

\chapter{Introduction}

\epigraph{Everyone in a complex system has a slightly different interpretation. The more interpretations we gather, the easier it becomes to gain a sense of the whole.}{Margaret J. Wheatley}


%% - Emerging of Information technology 
How is it possible for the nail size chip in our phone to understand us speaking or to recognise our face among thousands of people? 
It seems like a kind of magic that used to come form science fiction and in less than a decade has became a contingent reality: every picture of all phones is automatically tagged, any face recognised and all conversations dictated. 
It's not a very surprising fact though that, as a result, many people are actually rising concerns about this growth of technology and even governments are tightening regulations to guarantee the privacy rights against the economic use of people personal information.
~
%% - We can profit of IT for scientific experiment
Nonetheless, aside this new information business, those novel solutions that are mainly pushed by private interests are also very commonly released with open-source licences, thus being available to the scientific community. The idea is that vendors take advantage from a good code quality and the proliferation of new ideas within the wide pool of open-source developers, and at the same time the community get access to a cheaper hardware and very powerful software tools that can be adapted to any specific need.
~
%% - Data over algorithm -> ML over AI
This new era, led by information theory, is revealing not emerging from complicate mathematical models but from the data itself,0 joined with lower cost per computational operation in terms of energy consumption and components price. %TODO: sistemare %
The predominant role of data is on one side promoting the big infrastructures where huge amount of signals must be collated though efficient databases and promptly passed through analysis algorithms, on the other side it stresses the need of an efficient design of hardware that must be able to adapt to changes of algorithm and data dimensionality. This transition from algorithms to big-data analysis matches the shifting interest from the general \ac{AI} approach to the \ac{ML}\footnote{Machine learning is a subset of \acs{AI}. As from the definition by Tom Mitchell, author of the book "Machine Learning": A computer program is said to learn from \textit{experience} $E$ with respect to some class of \textit{tasks} $T$ and \textit{performance} measure $P$ if its performance at \textit{tasks} in $T$, as measured by $P$, improves with experience $E$.}, while from the algorithmic perspective it established the preeminent role of \ac{ANN}.

%% - Experiments may use hard correlated signals
The classical scientific approach is to make hypotheses on an evidence insulating repeatable experimental values and to provide a model that is able to reproduce those values. But it is not uncommon that within a complex experiment many different systems coexist, each with its own representation model, exposing a non trivial connection each others. There is also the intuitive perception that gathering more and more information about a phenomenon gives a more accurate representation, but this requires to manage all the possible non-linear correlations coming from the different nature of signals.
~
%% - NN universal approximation
On the other hand it can be proved that any continuous convex function of n-dimensional input variables can be effectively approximated with a n+1 width neural network~\cite{csaji2001approximation}\cite{hanin2017universal}.
%TODO: ESPANDERE %
~
%% - deep learning
The actual turning point in ML was the NIPS'12 publishing of AlexNet~\cite{NIPS2012_4824}, a massive GPU trained network that won the ImageNet Large Scale Visual Recognition Challenge\footnote{The ILSVRC (\url{http://www.image-net.org/challenges/LSVRC/}) evaluates algorithms for object detection and image classification at large scale. One high level motivation is to allow researchers to compare progress in detection across a wider variety of objects, taking advantage of the quite expensive labeling effort.} achieving a top-5 error 10.8\% lower than the runner up. 
The original paper's primary result was that the depth of the model was essential for its high performance opening the era of Deep Learning~\cite{Goodfellow-et-al-2016}.

%% - this thesis data integration using deep learning
A complex and heterogeneous experimental system could then benefit of this methodology in many ways: the possibility to integrate of many different kinds of signals, the robustness to react on missing or corrupted data in case a portion of the sensory system fails, and the flexibility to adapt to changes on both the parameters or the environment variables of the experiment.
~
This thesis dissertation will present a set of tools that exploit such data integration, using the deep learning techniques from both the hardware and software perspective, applied to a possible scenario of controlling the magnetic confinement of a nuclear fusion plasma. 
% TODO: add.. in addition we will present the hardware implementation suitable to achieve the task
%
%% this is indeed a complex experimental setup applied to RFX-mod
This is, indeed, a very complex task other than one of the main challenges of this century; and in this context the \textit{RFX~consortium} is hosting RFX-mod, one of the most controllable fusion experiments currently active in fusion research~\cite{SONATO2003161}\cite{doi:10.1063/1.4806765}.
Recently a further update begun that will bring to RFX-mod2, improving even more the machine responsiveness together with the renew of the sensory system; in this environment fits this research application with a specific design proposal that integrates a \ac{ML} based control loop to the overall chain of data acquisition and control.

%TODO: REVIEW DOPO AVER SCRITTO INTRO%
In the next section a brief introduction to the problem of magnetic confinement will be presented followed by a description of the intuitive principles that justify the application of this kind of control.




\section{magnetic confined nuclear fusion}
% FUSION %
The referred nuclear process is the exothermic reaction that happens between two light atom nuclei that fuse together. A fusion process that produces a transmutation to a new nucleus lighter than iron-56 or nickel-62 generally yields a release of energy. These elements have the smallest mass per nucleon and the largest binding energy; so the fusion of light nuclei toward, as well as fission of elements beyond, these elements releases the energy retained, resulting in a exothermic reaction.  This means that, with the purpose of extracting energy, the lighter elements (hydrogen and helium) are in general more likely to be the fusible fuel, while the heavier elements (uranium, thorium and plutonium) are more fissionable fuel. 
Figure~\ref{fig:binding} shows the binding energy per nucleon as a function of the atomic mass number. The exceptional situation of 4 He is shown. Fusion processes ending in He show the largest exothermal response; the burning of 1 H to 4 He releases most of the energy available to fusion. The maximum of the binding energy is with 56 Fe.

\begin{figure}[ht!]
\includegraphics[height=0.25\textwidth]{img/binding_energy.jpg} \centering
\includegraphics[height=0.25\textwidth]{img/abundance.png} \centering
\caption{binding energy per nucleon for each element atomic number. }
\label{binding}
\end{figure}

% FUELS and abundancy of elements %
The universe as forged by the big-bang was initially mainly formed by leptons along with simple elements: protons, deuterons, and a little quantity of helium and lithium. All the remaining set of the 92 known elements, from carbon up to uranium are afterwards the expression of the self-organisation of the universe - the formation of galaxies where the stars burns hydrogen as breeder material for new elements - enforced by gravity\footnote{ Lithium, Beryllium and Boron are relatively rare in cosmos since they had little time to form in the first expansion and are not directly synthesized by stars; their nuclei are actually destroyed within the star's core, however a small quantity can be produced by break-up of heavier elements in interstellar dust, as a result of impact by cosmic rays\cite{LiBeB_syntesis}}.

% Fusion and Fission in cosmos %
The production of heavy fission fuels by nucleosynthesis is generally ascribed to extreme astrophysical events like supernovas that are able to reach enough energy to fuse nuclei into elements heavier than iron; when we use nuclear power coming from fission process we are actually recovering that amount energy that are relatively rare to find in cosmic matter. 
% 
On the other hand, as stated, the most common spread process in the universe is the burning of Hydrogen in the so called \textit{pp} chain reaction, firstly described by H.Bethe in 1939. 
In the core of a star, gravity produces high density and high temperature. The density of gas in the core of our sun is 160 g/cm3, much higher than the densest metal, and the temperature is about $15\times10^6K$ ( $\simeq 1.5 KeV$ ). Under these extreme conditions the kinetic energy of particles and the continuous collisions disentangle all electrons from nucleons and the matter is in form of the plasma; protons (H-1) react with other protons to make deuterium nuclei (H-2) and positrons. The deuterium nuclei can merge to form a helium nuclei (He-4), or they can interact with other protons to make another isotope of helium (He-3). Two He-3 nuclei can fuse to make a nucleus of an unstable beryllium nucleus (Be-6) that breaks apart to give He-4 and two protons. Energy is released at each step.

In this sense the Hydrogen fusion is the most common way to obtain energy in the cosmos, and it is not by chance that the majority of the earth available energy resources are actually coming from the sun.

% Fusion in earth %
Starting form the Oliphant, Harteck and Rutherford experiments, that in 1934 provided for the first time the proof of transmutation accelerating Hydrogen~\cite{1934RSPSA.144..692O}, the attention quickly moved toward the plasma physics research by means of thermal excitation with the proposal of the \textit{tokamak} as a magnetic confinement device in 1950 by the soviet physicists Igor Tamm and Andrei Sakharnov~\cite{}. 
%
% Cross section
% RR 
The fusion process between positively charged nuclei has to overcome the Coulomb repulsion. The potential energy at a distance of the proton diameter is about 0.6 MeV. But nature eases fusion because the particle, the proton, adopts in this interaction the characteristics of a wave, which allows it to leak into the energetically prohibited zone and to finally tunnel through the Coulomb barrier into the deep bound states of the newly formed nucleus. The potential energy at the distance of the de~Broglie wave length is about two orders of magnitude lower than the one within the range of the nuclear forces. The fusion yield depends on the density of the particles, their relative velocity and the nuclear interaction in the form of the fusion cross-section. 
The competing interaction processes are Coulomb collisions between charged particles. As the Coulomb collision rate is large compared to the fusion rate the proton assembly (as well as the one of the electrons) can be described by a Maxwellian distribution. The governing kinetic parameter is the temperature rather than the individual particle energy in the assembly interactions — scattering and fusion. This is the reason why we are speaking of controlled thermo-nuclear fusion.

% Lawson and why tokamak
The idea follows the simple principle that we need to produce fusion events at a rate that the system can, at least, virtually self sustain the reaction. 
\cite{Wagner.Friedrich:magnetic.confinement.intro}
The rate of fusion power normalized by the amount of external power needed is typically defined as the \textbf{Q} factor; 
% cross section rates and selection of thermal reaction
The selection of the proper reaction is guided, however, by searching for the highest reaction rate providing the highest fusion yield. This condition is fulfilled by fusing deuterium $D=\isotope{2,H}^+$, the heavy isotope of hydrogen, with tritium $T=\isotope{3,H}^+$, the super-heavy one.
% give temperature profiles for breakeaven and ignition


%
% magnetic confinement and MHD
The magnetic confinement relates to the fact that in those working conditions of temperature and pressure the reactants are completely ionized and all the particles subjected to Lorenz's force.
%
% RR
In the post-war era, a number of researchers began considering different ways to confine a plasma. George Paget Thomson of Imperial College London proposed a system now known as z-pinch, which runs a current through the plasma. Due to the Lorentz force, this current creates a magnetic field that pulls the plasma in on itself, keeping it away from the walls of the reactor. This eliminates the need for magnets on the outside, avoiding the problem Fermi noted. Various teams in the UK had built a number of small experimental devices using this technique by the late 1940s.

%% LINEAR (CYL) CONFIGURATION

% - theta pinch can not be bent in a torus.
% Theta pinch equilibrium is easy to find ... see MHD

% - solution is the z-pinch 
% the MHD equilibrium can be solved by the Bennet relation
% In contrast to the Θ-pinch, for a Z-pinch it is the tension force and not the magnetic pressure gradient that provides radial confinement of the plasma
% it is not stable but can be bent in a torus

% screw pinch - general combination
% Though the momentum equation is non-linear, the Θ-pinch and Z-pinch forces add as a linear superposition, a consequence of the high degree of symmetry.
% the stability is granted by theta component and zeta brings the possibility to realize a toroidal shape.



%
Nowadays several families of magnetic configurations characterize different kind of devices; the structural design of each fusion machine reflects the arrangement chosen for the internal field. the best topological configuration seems to be the torus. The main reason of this choice is due to the fact that the wanted pinch effect of the magnetic field is created by a current that flows thorough the plasma and can be closed in a loop by the toroidal configuration; in this case we also avoid any loss of particles that a linear configuration would suffer. The torus is the only also the only compact connected domain where it is possible to define a continuous vector field without critical points,  where field lines are usually winded around the the shape with helical paths. 
In this case the magnetic components can be defined in a 
%


In a magnetic field charged particles move in a helix  F = q · v × B.  q is the charge, v the individual particle velocity, B the magnetic field. In the direction perpendicular to the field, the excursion of the particle is limited to the Larmor radius $\rho_L$ . Because of the specific form of the Lorentz force as vector product of velocity and field, the particle momentum parallel to the field is not changed. Therefore, in a homogeneous field, magnetic confinement is insufficient because it applies to the perpendicular velocity component only. One has to involve and to accept systems with field inhomogeneity to improve confinement. Magnetic confinement systems differ in the way they cope with the confinement parallel to the field thus defining classes of confinement systems.

% In linear devices — the first confinement class — parallel confinement is improved utilizing the mirror effect. The basis of the mirror effect is the magnetic moment μ = 12 mv ⊥ /B of a magnetised charged particle, which is a constant of motion.
% An electron or ion, which is thermally agitated with velocity v = (v ⊥ , v  ) and which moves
% from a zone of low field B min into one with higher field increases its perpendicular
% velocity v ⊥ as a consequence of μ = const. On the other hand, the kinetic energy of the
% plasma species is also a constant of motion with the corollary that the parallel velocity
% v  decreases. If the field is sufficiently large v  → 0. At this field, the mirror field B max ,
% the particle comes to a full stop. Thereafter, it is accelerated back to the low-field
% zone by the force −μ · grad  B. If the field system is built in a symmetric way, the
% game repeats itself at the other side with the outcome that the mirror effect causes the
% particles to bounce between two mirror points and to stay in the neighbourhood of the
% field minimum thus improving confinement. In an actual mirror machine, characterized
% by B min and B max , two classes of particles, discriminated by v  /v, have to be considered.
% If this velocity ratio is small, a particle is reflected by the magnetic mirrors as described
% above and represents a trapped one. Otherwise, it is slowed down along its trajectory
% toward higher field but does not reach the point with v  /v = 0. These particles escape
% the mirror. The boundary is given by v ⊥
% /v 2 = B min /B max specifying a loss-cone in
% phase space spanned by v ⊥ and v  as coordinates.

% The confinement of simple mirror machines is insufficient because the transparency
% of the mirror is too large. Still, we continue describing the confinement situation of the
% mirror because the concepts help us to understand the physics of more relevant confine-
% ment systems. Up to now, we have ignored collisions between the plasma species. Of
% relevance are the collisions in phase space of the trapped particles occupying the low-field
% zone scattered into the empty loss cone. Because of their higher velocity, electrons are
% scattered more frequently into the loss cone than the ions. As a consequence, the plasma
% is polarised and charges up positively. The formation of an electric field keeps back
% the escaping electrons by electrostatic means and accelerates the ions. For a transport
% equilibrium, the so-called ambipolar electric field enforces flux equality Γ e = Γ i between
% the differently charged plasma species of different masses and different kinetic properties.
% A confinement concept avoiding mirror losses is toroidal confinement realised in a
% plasma ring as shown in fig. 4. In this case, the magnetic coils are arranged in a closed
% ring thus avoiding end-losses. This geometry defines the class of toroidal confinement
% systems. An unavoidable feature is the inhomogeneity of the toroidal field with a higher
% field closer to the vertical symmetry axis. The field gradient points to the symmetry axis.
% This field inhomogeneity leads to rather unpleasant consequences: The Larmor orbits of
% the charged particles are not closed causing drifts. Ions and electrons drift parallel to the
% symmetry axis and leave their magnetic field line. The drift of electrons and ions is in
% opposite direction leading to an additional vertical electric field. A second parasitic drift
% appears now in the crossed-field arrangement of vertical electric and toroidal magnetic
% field. Like in the Hall effect an E × B drift appears perpendicular to both E and B tor
% directions, which causes the plasma torus to expand radially. No force equilibrium is
% established.

% The saving idea was to introduce rotational transform. A second field component
% B pol is introduced with a perpendicular (poloidal) component. A field line with the
% components (B tor , B pol ) winds around the torus in a helix (see fig. 4) and does no longer
% stay in a plane rather maps out a toroidal surface. The vertical charge separation is
% avoided because the up-down sides of the torus are short-circuited by the helical field
% lines connecting these regions. Macroscopic radial equilibrium can be provided.
% The essence of magnetic confinement is the development of a nested system of toroids
% with field lines, which encircle the tori toroidally and poloidally and with a field line
% density, which is higher at the inside than the outside (torus effect). The nested toroids,
% called flux surfaces (see fig. 4), are magnetically isolated and not connected by a field
% line with a net radial component.
% The way the poloidal field component is produced defines confinement classes within
% toroidal systems. The simplest one is using the potential of a plasma to carry a current. A
% ring current flowing inside the plasma, the plasma current I p , produces the poloidal field
% component B pol . As the temperature of the electrons, defining the electrical conductivity
% of a plasma, is highest in the plasma core, the current density profile j(r) peaks there.
% Such systems are called tokamaks [7] and are a Russian invention of the ’50s of last
% century( 8 ). Figure 5 shows the principal set-up of a tokamak. The major attraction of
% the tokamak is that I p can be produced by a pulsed transformer whereas the plasma
% ring surrounding a central primary coil acts as secondary winding. In tokamaks, the
% ratio of B tor /B pol > 1. Another concept with plasma currents is the reversed field pinch
% (RFP) with B tor /B pol ∼ 1 [8]. Tokamaks and RFPs belong to the category of internal
% confinement systems because part of the confining magnetic field is produced by currents
% flowing inside the plasma. 

%An alternative way is to produce the poloidal field like the toroidal one — by external
% coils. In this case, the coils have to wind helically around the torus. The field composition
% reminds of a multipole arrangement with coils being first helically twisted and then bent
% into a torus. Depending on the current direction inside the helical coils — unipolar or
% bi-polar — we speak of heliotrons/torsatrons or of stellarators. Both belong to the class
% of helical systems. Stellarators are a US invention of the ’50s. Figure 6 shows a sketch
% of a classical l = 2 stellarator with helical coils. Helical systems belong to the category
% of external confinement.
% Toroidal systems share common descriptors. In the simplest case with circular poloidal
% cross-section the torus geometry is defined by major radius R and minor radius a (see
% fig. 4). The ratio A = R/a is called aspect ratio. The rotational transform ι of a flux
% surface is defined by the winding law of the field lines given by ι/2π = AB pol /B tor . RFPs
% have a larger ι than tokamaks. In case of helical systems, B pol is determined in a complex
% form by the external helical coils system, which is described by the number of coils e.g.
% 3 heliotron coils with unidirectional current yielding an l = 3 heliotron. In this case the
% poloidal cross-section has the shape of a triangle. In case of a stellarator with 4 helical
% coils and bi-polar currents, we have 2 pairs of coils and an l = 2 stellarator (see fig. 6).
% Its poloidal cross-section is elliptical. Another important stellarator descriptor is the coil
% pitch. The coil winding laws are such that the coils ultimately close. The field pattern
% of helical systems is periodic in toroidal direction. The toroidal periods can be m = 5
% (Wendelstein 7-A, Germany [3]) or in case of tight windings e.g. m = 10 (Large Helical
% Device, LHD, Japan [3]).
% Simple tokamaks have a circular cross-section. Equilibria at higher plasma currents
% are possible when the cross-section is elongated or even has triangular shape. Modern
% tokamaks benefit from the higher current:




\section{A new approach based on machine learning}
\cite{Goodfellow-et-al-2016}

% From the statistical perspective, supervised neural networks are nothing more
% than nonlinear curve-fitting devices. Curve fitting is not a trivial task however.
% The effective complexity of an interpolating model is of crucial importance,
% as illustrated in figure 44.5. Consider a control parameter that influences the
% complexity of a model, for example a regularization constant α (weight decay
% parameter). As the control parameter is varied to increase the complexity of
% the model (descending from figure 44.5a–c and going from left to right across
% figure 44.5d), the best fit to the training data that the model can achieve
% becomes increasingly good. However, the empirical performance of the model,
% the test error, first decreases then increases again. An over-complex model
% overfits the data and generalizes poorly. This problem may also complicate
% the choice of architecture in a multilayer perceptron, the radius of the basis
% functions in a radial basis function network, and the choice of the input vari-
% ables themselves in any multidimensional regression problem. Finding values
% for model control parameters that are appropriate for the data is therefore an
% important and non-trivial problem.

% The overfitting problem can be solved by using a Bayesian approach to
% control model complexity.
% If we give a probabilistic interpretation to the model, then we can evaluate
% the evidence for alternative values of the control parameters. As was explained
% in Chapter 28, over-complex models turn out to be less probable, and the
% evidence P (Data | Control Parameters) can be used as an objective function
% for optimization of model control parameters (figure 44.5e). The setting of α
% that maximizes the evidence is displayed in figure 44.5b.
% Bayesian optimization of model control parameters has four important ad-
% vantages. (1) No ‘test set’ or ‘validation set’ is involved, so all available training
% data can be devoted to both model fitting and model comparison. (2) Reg-
% ularization constants can be optimized on-line, i.e., simultaneously with the
% optimization of ordinary model parameters. (3) The Bayesian objective func-
% tion is not noisy, in contrast to a cross-validation measure. (4) The gradient of
% the evidence with respect to the control parameters can be evaluated, making
% it possible to simultaneously optimize a large number of control parameters.
% Probabilistic modelling also handles uncertainty in a natural manner. It
% offers a unique prescription, marginalization, for incorporating uncertainty
% about parameters into predictions; this procedure yields better predictions, as
% we saw in Chapter 41. Figure 44.6 shows error bars on the predictions of a
% trained neural network.

% As was mentioned in Chapter 41, Bayesian inference for multilayer networks
% may be implemented by Monte Carlo sampling, or by deterministic methods
% employing Gaussian approximations (Neal, 1996; MacKay, 1992c).
% Within the Bayesian framework for data modelling, it is easy to improve
% our probabilistic models. For example, if we believe that some input variables
% in a problem may be irrelevant to the predicted quantity, but we don’t know
% which, we can define a new model with multiple hyperparameters that captures
% the idea of uncertain input variable relevance (MacKay, 1994b; Neal, 1996;
% MacKay, 1995b); these models then infer automatically from the data which
% are the relevant input variables for a problem.



% supervised and unsupervised learning.
% changing the perspective from algorithm to data representation.

% it is not the lose of the model
% - model can be applied to add information ( fourier for example )
% - model can be added to confront the dimensionality of the latent space

% no free launch theorem
% course of dimensionality


% introduction to the bicycle example 
\section{How to ride the bicycle \\ \small{ equilibrium through latent space shaping }}
\cite{rideabike_nature_2016}


\section{Mildstone \\ \small{ a framework for learning within the machine }}
% explain why it is needed ( of worth ) the complete redesign for a new acquisition chain.
% explain why we need a framework for that.
% tensorflow
% MDSplus, MARTe
% ...

\section{Structure of the document}