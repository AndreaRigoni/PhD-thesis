
Per collegare l'azione del controllo esterno agli effetti sulla
superficie di plasma mediante metodo magnetoidrodinamico è necessario
conoscere il valore puntuale di campo e del suo differenziale; per
semplificare la trattazione analitica è stato proposto in \cite{pizz81}
un modello di RFP cilindrico lineare\footnote{in appendice è riportata
le formulazione per gli operatori differenziali in coordinate
cilindriche utilizzati nel modello lineare.}. Lo schema considerato
prevede una zona di quasi vuoto che separa la colonna di plasma dalla
parete.  I sensori, coassiali alle bobine di controllo, possono ricavare
misure di campo radiale, poloidale e toroidale, interne ed esterne (in
seguito rispettivamente $b^r,b^{p\pm},b^{t\pm}$).  Sono considerate
evoluzioni temporali esponenziali per le grandezze perturbate, ovvero
caratterizzate da un andamento temporale del tipo $e^{\gamma t}$. Il
modello sfrutta la posizione detta di \emph{thin shell} che considera la
parete della camera a vuoto come un conduttore omogeneo sottile per
ottenere una semplificazione della analisi teoretica.  Lo schema
generale del controllo prevede una composizione in componenti di Fourier
della perturbazione rilevata, di seguito un amplificatore a controllo
digitale genera un segnale di guadagno per riequilibrare i modi
instabili, infine il valore del segnale di controllo e antitrasformato e
applicato direttamente agli amplificatori.  Si vuole quindi trovare un
legame tra l'azione del controllo e l'effetto prodotto in termini di
campo, rivelato dai sensori posti sulla parete della camera,
interponendo un modello fisico che agisca da cuscinetto.  La reazione
del modello fisico prevede una dipendenza spaziale, che comporta un
ampliamento in frequenza dello spettro nella analisi modale, e la
dipendenza temporale, che si esplica nel comportamento della parete alla
penetrazione del campo.

\subsection{Estensione Armonica}

L'applicazione del campo di controllo discretizzato alle posizioni delle
bobine attive, lungo la direzione poloidale e toroidale della camera,
introduce una inevitabile dispersione in frequenza.
Si cerca di trovare un'espressione spettrale completa dell'effetto di
una generica componente armonica nella corrente di controllo $I\mn$.

L'espressione cercata non può prescindere dalla geometria delle bobine
oltre che dalla loro dislocazione.  E' importante ricordare che il
modello dipende da una approssimazione cilindrica di raggio $a$ e
periodicità lungo l'asse $f(r,\vartheta,z) = f(r,\vartheta,z+2\pi
R_0)$. Nel modello sono considerati due strati di bobine: la griglia dei
sensori, capaci di misurare componenti di campo in prossimità della
superficie interna della parete, e la griglia degli attuatori alla
distanza $r_f > r_w$. Le griglie sono formate da serie di $M_c \times
N_c$ solenoidi rispettivamente lungo la direzione poloidale e toroidale.
E' inoltre possibile senza commettere un errore aprezzabile, assumere
che tali solenoidi siano distribuiti ad angolo regolare e presentino
tutti la stessa dimensione.

Come analizzato in \cite{pizz_81} si considera valida la posizione di
\emph{thin shell} che sarà discussa nel prossimo paragrafo, è quindi
possibile esprimere la densità di corrente attraverso una funzione di
flusso.  La corrente negli avvolgimenti di feedback è rappresentata
attraverso una funzione potenziale $J^f$ così che la densità risulti
$\nabla J^f \times \hat{r}$. Si introduce inoltre per il generico
attuatore la funzione di sagoma $f(\vartheta,\varphi)$ che risulta
unitaria allinterno del profilo della bobina e nulla altrove.  La somma
dei contributi presenta evidentemente una dislocazione discreta del
potenziale:
\begin{equation}
 \label{eq:potenziale}
 J^f(\vartheta,\varphi) = I\mn \sum_{j=0}^{M-1} \sum_{k=0}^{N-1}
  e^{ i (m\vartheta_j+n\varphi_k)} f(\vartheta - \vartheta_j,\varphi-\varphi_k)
\end{equation}

dove $\vartheta_j = j2\pi/M$ e $\varphi_k = k2\pi/N$

la trasformata diventa:




\begin{align}
 FT(J^f) &= J^f\mnp = \frac{1}{(2\pi)^2} \iint_{-\infty}^{+\infty} J^f(\vartheta,\varphi)
  e^{-im'\vartheta} e^{-in'\varphi} d\vartheta d\varphi \nonumber \\
 &=\frac{I\mn}{(2\pi)^2}\sum_{j=0}^{M-1} \sum_{k=0}^{N-1}
 \iint_{-\infty}^{+\infty} f(\vartheta - \vartheta_j,\varphi-\varphi_k)
 e^{i(m\vartheta_j+n\varphi_k)} e^{-i(m'\vartheta + n'\varphi)}
 d\vartheta d\varphi  \nonumber
\end{align}
aggiungendo e sottraendo un termine si ottiene:
\begin{align}
 m\vartheta_j - m'\vartheta + m'\vartheta_j - m'\vartheta_j &=
 (m-m')\vartheta_j - m'(\vartheta - \vartheta_j) \nonumber \\
 n\varphi_k - n'\varphi + n'\varphi_k - n'\varphi_k &=
 (n-n')\varphi_k - n'(\varphi - \varphi_k) \nonumber
\end{align}
la funzione diviene:
\begin{align}
 \label{eq:dimostrazione_sideb}
= \frac{I\mn}{(2\pi)^2} \sum_{j=0}^{M-1}\sum_{k=0}^{N-1}
 \iint_{-\infty}^{+\infty} f(\vartheta - \vartheta_j,\varphi-\varphi_k)
 e^{-im'(\vartheta-\vartheta_j)}e^{-in'(\varphi-\varphi_k)}  d\vartheta
 d\varphi \nonumber \\
 \times \quad e^{i(m-m')\vartheta_j}e^{i(n-n')\varphi_k} \nonumber \\
 = I\mn F_{m',n'} \frac1{M} \sum_{j=0}^{M-1} \exp\left(2\pi i
 \frac{m-m'}{M}j \right) \cdot 
 \frac1{N} \sum_{k=0}^{N-1} \exp\left(2\pi i \frac{n-n'}{N}k \right)
\end{align}

dove si sono sostituiti i valori $\vartheta_j = j2\pi/M$ e
$\varphi_k=k2\pi/N$. La funzione di forma trasformata è riassunta in:

\begin{equation}
 F\mn = \frac{MN}{(2\pi)^2}\iint_{-\infty}^{+\infty}
  f(\vartheta,\varphi)e^{-i(m\vartheta+n\varphi)} d\vartheta d\varphi
\end{equation}
In generale per un solenoide di forma rettangolare si assume il valore \cite{pizz81}
\begin{equation}
 \label{eq:shape_coils}
 F\mn = \frac{MN}{\pi^2}\frac{sin(m\Delta\vartheta_f/2)sin(n\Delta\varphi_f/2)}{mn}
\end{equation}

Per concludere, si è precedentemente posto la funzione di forma come
nulla fuori dall'area del solenoide e di valore unitario all'interno,
questo comporta che le somme in (\ref{eq:dimostrazione_sideb}) sono non
nulle per tutti i valori interi di $(m-m')/M$ e $(n-n')/N$; da cui
l'espressione per la generica componente armonica del potenziale:
\begin{equation}
 J\mn^f = F\mn \sum_{l=-\infty}^{+\infty}\sum_{p=-\infty}^{+\infty} I_{m+lM,n+pN}
\end{equation}
Si conclude che ogni modo $(m,n)$ esternamente applicato genera un
numero infinito di armoniche in $J^f$, ognuna delle quali separata da
multipli del numero di bobine del corrispondente asse.

Allo stesso modo anche il sistema di bobine che registrano il campo alla
parete introducono degli artefatti nello spettro, dovuti alla loro
dislocazione discreta. Come già delineato parlando della struttura
geometrica della macchina RFX, sensori e attuatori sono in numero uguale
e coassiali, cosicchè é semplice intuire come le bande laterali che
generano siano per costruzione uguali. Si assume quindi di introdurre
l'effetto di estensione armonica dei sensori con $S\mn$ \cite{pizz_81}
\begin{equation}
 \label{eq:shape_sensori}
 S\mn = \frac{sin(m\Delta\vartheta_s/2)}{m\Delta\vartheta_s/2}
\frac{sin(n\Delta\varphi_s/2)}{n\Delta\varphi_s/2}
\end{equation}
Tale effetto sagoma le componenti del campo che investe i sensori
producendo il profilo del campo rivelato:
\begin{equation}
 B\mn = \sum_l \sum_p S_{m+lM,n+pN} b_{m+lM,n+pN}
\end{equation}

Un metodo alternativo ma ugualmente efficace, per analizzare l'estensione
armonica dei modi introdotti dal controllo, consiste nel considerare la
teoria dei segnali imponendo la disposizione spaziale delle bobine come
la successione dei campioni di un segnale bidimensionale. Una soluzione
tipicamente adottata nell'elaborazione delle immagini.

L'analisi dei segnali è convenientemente studiata attraverso una
\emph{teoria unificata} \cite{cariolaro} dove la generica trasformata
di Fourier presenta sempre nucleo esponenziale ma utilizza la
formulazione di Haar per l'integrazione:
\begin{equation}
 F({\bf v}) = \int_I d{\bf u} f({\bf u})e^{i2\pi{\bf u}^T{\bf v}} \quad \quad
  \begin{array}{r}
   {\bf u} \in I \\
   {\bf v} \in \hat I
  \end{array}
\end{equation}
Il dominio di definizione $I$ e il corrispondente dominio duale $\hat I$
sono costituiti da gruppi regolari genericamente espressi in forma
quoziente: $$ I = \frac{I_0}{S} \longleftrightarrow \hat{I} =
\frac{S^\star}{I_0^\star} $$ dove il gruppo denominatore indica la
periodicità, il numeratore la discretizzazione, e l'opratore $\star$
indica il gruppo reciproco come: $$J^\star = \{{\bf v}| {\bf
u^Tv}\in\mathbb{Z} ,\quad {\bf u}\in J \}$$ Questa digressione puramente
teorica è utile, poichè si può pensare alla distribuzione delle
\emph{saddle coils} come ad un reticolo bidimensionale ortogonale
$\mathbb{Z}(d_1,d_2)$, dove $d_1 = 2\pi/M$ e $d_2 = 2\pi/N$. In questo
modo il segnale di controllo diventa bidimensionale discreto e periodico
di $M\times N$ campioni mentre la trasformata, come integrale di Haar,
diviene la usuale trasformata discreta (DFT) e porge quindi un segnale
definito nel dominio duale con la stessa periodicità e
discretizzazione. Successivamente, si applica una interpolazione del
dominio tramite convoluzione del segnale con la funzione di sagoma
$f(\vartheta,\varphi)$; il dominio di definizione presenta comunque la
stessa periodicità ma diviene una funzione continua. Nel dominio duale
il prodotto di convoluzione è invece un semplice filtraggio che modula
in ampiezza le infinite armoniche eliminandone la periodicità.




\subsection{Matching Procedure}
L' andamento completo del campo, tipico problema di valori al contorno
  \cite{jackson}, è formalmente risolubile mediante equazioni di Green,
  ma tale metodo non sempre risulta agevole
  analiticamente. L'alternativa spesso utilizzata è un approccio che
  parte direttamente dalla formulazione differenziale e usa uno sviluppo
  in funzioni ortogonali.  Nella regione di vuoto tra il plasma e la
  parete ($r<r_w$) e tra la parete e le bobine di controllo
  ($r_w<r<r_f$) vale l'equazione di Poisson, e il campo è quindi
  misurabile attraverso un gradiente di potenziale magnetico:
\begin{align}
 \label{eq:Poisson}
 \nabla^2\Phi = 0 \\
 \label{eq:grad_phi}
 \vettore{b} = \nabla\Phi
\end{align}
Lo sviluppo della (\ref{eq:Poisson}), proposto in appendice, è
realizzato mediante separazione ortogonale delle variabili e nel caso
cilindrico comporta un andamento del potenziale caratterizzato dalla
equazione differenziale di Bessel e di seguito descritto tramite
Funzioni di Bessel modificate (MBF):
\begin{equation}
 \Phi_r = A \cdot I_m(|n|\epsilon_r) + B \cdot K_m(|n|\epsilon_r)
\end{equation}
in cui $\epsilon_r = r/R_0$.

Il campo radiale risultante è ovunque dato come combinazione delle
derivate parziali prime rispetto al raggio $I'_m(|n|\epsilon_r)$ e
$K'_m(|n|\epsilon_r)$. Uno schema del profilo del campo generato dalle
bobine a vuoto è disegnato in figura~\ref{fig:matchb}; si nota
immediatamente un massimo nelle vicinanze del punto di applicazione
della corrente di controllo e via via una attenuazione lungo la sezione
nelle due direzioni opposte.  Si dovranno quindi ricavare i valori dei
coefficienti A e B che realizzino tale profilo. Osservando l'andamento
asintotico delle MBF, graficate per alcuni valori del parametro m in
figura~\ref{fig:bessel}, è immediato rilevare che per l'annullarsi del campo
nelle due direzioni si dovranno eliminare le componenti non
asintoticamente nulle; ovvero porre $B = 0$ per $r<r_w$, successivamente
$A\neq0 ,B\neq0$ nell'intervallo $r\in[r_w,r_f]$, e infine $A=0$ per
$r>r_f$.

%%%%%%%%%%%%%%%%%%%%%%%%%%%%%FIGURA%%%%%%%%%%%%%%%%%%%%%%%%%%%%%%%%%%%%%
\begin{figure}[ht]
 \centering
 \includegraphics[ width=8cm ] {images/matchb.png}
 \caption{Schema del profilo di campo radiale}
\end{figure}
%%%%%%%%%%%%%%%%%%%%%%%%%%%%%FIGURA%%%%%%%%%%%%%%%%%%%%%%%%%%%%%%%%%%%%%

%%%%%%%%%%%%%%%%%%%%%%%%%%%%%FIGURA%%%%%%%%%%%%%%%%%%%%%%%%%%%%%%%%%%%%%
\begin{figure}[ht]
 \centering
 \includegraphics[ width=8cm ] {images/BesselI.png}
 \caption{Schema della funzione MBF I per valori di m=0,m=1,m=2,m=3}
\end{figure}
%%%%%%%%%%%%%%%%%%%%%%%%%%%%%FIGURA%%%%%%%%%%%%%%%%%%%%%%%%%%%%%%%%%%%%%

%%%%%%%%%%%%%%%%%%%%%%%%%%%%%FIGURA%%%%%%%%%%%%%%%%%%%%%%%%%%%%%%%%%%%%%
\begin{figure}[ht]
 \centering
 \includegraphics[ width=8cm ] {images/BesselK.png}
 \caption{Schema della funzione MBF K per valori di m=0,m=1,m=2,m=3}
\end{figure}
%%%%%%%%%%%%%%%%%%%%%%%%%%%%%FIGURA%%%%%%%%%%%%%%%%%%%%%%%%%%%%%%%%%%%%%




\begin{align}
 b_r &= A_1 \cdot I'_m(|n|\epsilon_r) && r < r_w \\
 b_r &= A_2 \cdot I'_m(|n|\epsilon_r) +  B_2 \cdot K'_m(|n|\epsilon_r) &&
 r_w < r < r_f \\
 b_r &= B_3 \cdot K'_m(|n|\epsilon_r) && r > r_f
\end{align}

Per quanto riguarda le condizioni al contorno sulla parete della camera
gli effetti dei modi RWM devono essere posti in relazione alla dinamica
temporale dell'impulso. La parete, infatti, può essere genericamente
associata ad una impedenza resistiva e induttiva con costante di tempo $L/R$.

Per confrontare i fattori di crescita con la risposta della parete si è
scelta la approssimazione \emph{''complete thin shell''}\footnote{
L'accezione ``completa'' si riferisce alla continuità del guscio;
eventuali tagli o buchi nella forma della parete, impedendo il recircolo
delle correnti indotte, modificano la differenza di campo nel salto di
parete riducendo l'attendibilità dell'aprossimazione.} che sottende alla
relazione:
\begin{equation}
 \frac{\delta_w}{r_w} \ll |\gamma|\tau_w \ll \frac{r_w}{\delta_w}
\end{equation} 
Il tempo caratteristico della parete in termini di $L/R$ equivalenti è
invece ricavabile tramite:
\begin{equation}
 \tau_w = \mu_0\sigma_w r_w \delta_w
\end{equation}
con $\delta_w$ spessore e $\sigma_w$ conducibilità del guscio.

In questo modo si assume irrilevante la densità di corrente radiale che
attraversa il sottile spessore $\delta_w$, così che le altre componenti
(e quindi il campo radiale) siano costanti e proporzionali alla
differenza tra campo trasversale interno ed esterno($b^{p\pm},b^{t\pm}$).

Con queste ipotesi possono essere dimostrate \cite{pizz70} le seguenti
relazioni che descrivono il profilo di campo radiale nell'attravesamento
di parete in $r_w$:

\begin{align}
 \label{eq:jump1}
 \jump{b_r}{r_w} &= 0  \\
 \label{eq:jump2}
 \jump{ \diffr{}(rb_r) }{r_w} &= \tau_w \gamma_{m,n}b_r
\end{align}

Nella regione di spazio dove è situato il solenoide attuatore del campo
di controllo si ipotizza un secondo guscio ideale che per il teorema di
equivalenza simula il campo generato attraverso una distribuzione di
correnti impresse sulla superficie. Considerando questo strato
immaginario di spessore $\delta_c$ e interessato dalla corrente
superficiale $\vettore{j} = (\rad{j},\pol{j},\tor{j})$ si identifica un
salto di campo radiale in $\epsilon_c = r_f/R_0$ proporzionale alla corrente
applicata:

\begin{equation}
 \jump{ \diffr{}(\rad{b}) }{r_c} = i \mu_0 \delta_c
  \left( \frac{m^2 + n^2 \epsilon_c^2}{m} \right) j_\varphi
\end{equation}

La densità superficiale $j_\varphi$ è direttamente dipendente dalla
corrente circolante nel solenoide attraverso un parametro geometrico
$\Sigma$:

\begin{equation}
 I_c = 2 \pi r_c \delta_c \Sigma j_\varphi
\end{equation}
Da cui la formula di interesse nel presente lavoro per $m=1$
\begin{equation}
 \label{eq:jump_coil}
 \jump{ \diffr{}(\rad{b}) }{r_f} = i \frac{\mu_0}{2\pi r_f^2 \Sigma}
  \left( {1 + n^2 \epsilon_c^2} \right) I_c
\end{equation}

L'andamento radiale atteso è schematizzato completamente in figura
[figura]: le relazioni al contorno (\ref{eq:jump1}), (\ref{eq:jump2}) e
(\ref{eq:jump_coil}) caratterizzano la continuità della funzione e la
derivata nei punti di giunzione dei tre intervalli. La procedura di
matching consiste allora nel determinare i paramentri $A_1,A_2,B_2,B_3$ per
ottenere un valore di campo $b_r$ lungo il profilo a fronte della
applicazione di una corrente $I_c$; questo valore sarà calcolato nel
campo a vuoto e utilizzato per dimensionare l'effetto del controllo
attivo sul campo disperso dalla parete.

%%%%%%%%%%%%%%%%%%%[figura]%%%%%%%%%%%%%%%%%%%%%%%%%%%%%%%%%%%%%
\begin{figure}[ht]
\caption{profilo del valore di potenziale per valori crescenti della
 coordinata radiale}
\end{figure}

Il campo generato dalla corrente nelle coils e dato da:

\begin{equation}
 \label{eq:bf_coils}
 b^f\mn = -\frac{\mu_0}{R} n^2 \epsilon_f \besKKne{f} \besIIne{f} J^f\mn
  \equiv \frac{\mu_0}{\pi a}c\mn J^f\mn
\end{equation}
[spiegare il termine $\mu_0/\pi a$]
Con $J^f\mn$ si è precedentemente definito il potenziale della
distribuzione delle correnti di modo $m,n$ localizzate sulla superficie
immaginaria in $r_f$ dove:
\begin{equation}
 c\mn = -\pi n^2 \epsilon_a \epsilon_f \besKKne{f} \besIIne{f}
\end{equation}

A questo punto, determinato il campo impresso dalle sorgenti, si ottiene
il valore nel punto di matching attraverso la seguente relazione che
riassume tutta la procedura.

\begin{equation}
 \label{eq:bw_coils}
 \{b^r\mn,b^{p\pm}\mn, b^{t\pm}\mn \}_{sens} =
  \{1,a^{p\pm}\mn,a^{t\pm}\mn\} M\mn b^f\mn
\end{equation}

Il valore $M\mn$ è un coefficiente di mutua induttanza per la relativa
frequenza spettrale, esso è ricavato risolvendo il problema
elettrostatico tramite le sopraindicarte condizioni al contorno. Una
dimostrazione rigorosa per la soluzione seguente è proposta in
appendice:

\begin{equation}
 \label{eq:emme}
  M\mn = - \frac{1}{2\tau_w(s-\gamma\mn)} \frac{1+m^2/n^2\epsilon_w^2}{\besKKne{w}\besIIne{f}}
\end{equation}

con uniformità di simboli si è espressa con $\gamma\mn$ il tasso di
crescita del modo $(m,n)$ analizzato. Si nota immediatamente che la
funzione $M\mn(s)$ presenta un polo corrispondente al tasso di
crescita del RWM corrispondente $s=\gamma\mn$.

I simboli $a^{p\pm},a^{t\pm}$ rendono il rapporto interno ed esterno di
campo poloidale e toroidale rispetto al valore radiale calcolato, ed
equivalgono a:

\begin{align}
 a^{\{p^-,t^-\}}\mn &= s_0\{m,n\epsilon_w\}\frac{d\mn - 1 +
 2\tau_w\gamma\mn}{m^2+n^2\epsilon^2_w} \\
 a^{\{p^+,t^+\}}\mn &= s_0\{m,n\epsilon_w\}\frac{d\mn - 1 +
  2\tau_w(\gamma\mn -s)}{m^2+n^2\epsilon^2_w} 
\end{align}
dove $s_0 = \pm1$ è un termine di segno introdotto per eliminare il
termine immaginario nel campo poloidale, si esegue infatti uno shift di
fase di 90° nel segnale. E' sufficiente porre tale valore a $sign(m)$
oppure $sign(n)$ rispettivamente per i sensori poloidali e toroidali.

Il termine
\begin{equation}
 d\mn = 1 - \left( 1+\frac{m^2}{n^2\epsilon^2_w} \right) \frac{|n|\epsilon_w\besKne{w}}{\besKKne{w}}
\end{equation}
equivale a $rb'_r/b_r$ per il campo magnetico esterno $(r > r_w)$ in
assenza della corrente di feedback applicata.





\subsection{Controllo}


Si può quindi ricostruire una relazione conclusiva che tenga conto di
tutti i passaggi fino ad ora analizzati. Caratterizzando la relazione
iniziale (\ref{eq:bf_coils}) con il termine $J^f\mn$, che tiene conto della
generazione delle \emph{sidebands} e del fattore di forma $F\mn$ di
(\ref{eq:shape_coils}), si ottiene il valore del campo generato dalle
\emph{saddle coils}. Valutando opportunamente il problema elettrostatico
in geometria cilindrica e imponendo le condizioni al contorno della
approssimazione di \emph{thin shell} si ottiene il valore dello stesso
campo (\ref{eq:bw_coils}) attraverso un termine di mutua induttanza
$M\mn$. Infine si aggiunge il contributo spettrale $S\mn$ di
(\ref{eq:shape_sensori}), il termine dovuto ai sensori.  Componendo
tutti i termini si perviene ad una relazione di ingresso/uscita tra la
corrente di controllo e il segnale di campo ricostruito dai sensori per
il feedback. Per ogni modo $(m,n)$ eccitato dal controllo, con $|m|<M/2$
e $|n|<N/2$, si ottiene:

\begin{equation}
 B\mn = \sum_{m'=m+lM}\sum_{n'=n+pN} F\mnp S\mnp M\mnp
  \{1,a^{p\pm}\mnp,a^{t\pm}\mnp\} \frac{\mu_0}{\pi a} I\mnp
\end{equation}
e si considera quindi la funzione di trasferimento del sistema per la
singola componente $(m,n)$
\begin{equation}
 {P_1}\mn = \sum_{\mnp}\frac{\pi a B\mnp}{\mu_0 I\mnp}
\end{equation}
Ciò che si vuole ottenere è un valore di guadagno applicato alle bobine
che annulli il campo rilevato; questo si traduce un un regolatore a zero
caratterizzato dalla seguente funzione caratteristica nella trasformata
di Laplace dove si è indicato con $G\mn(s)$ lo spettro di una generica
funzione guadagno.
\begin{equation}
 1+G\mn(s){P_1}\mn(s) = 0
\end{equation}
[figura]

Ci limitiamo in questo caso a non analizzare sistemi di controllo ottimo
e valutare il solo caso di controllo proporzionale con G indipendente
dal tempo.
Il sistema di figura può essere facilmente rappresentato da un sistema
di stato.
\begin{equation}
 \left\{
  \begin{array}{l}
   \dot{x}(t) = Ax(t) + Bu(t) \\
   y(t) = Cx(t)
  \end{array}
 \right.
\end{equation}
Considerando la reazione statica dall'uscita, si dimostra facilmente che
il sistema può essere ricondotto ad una tipica retroazione dallo stato;
infatti dalla legge di controllo del tipo:
\begin{equation}
 v(t) = Gy(t) + u(t)
\end{equation}
si ricava facilmente 
\begin{equation}
v(t) = GCx(t) + u(t)
\end{equation} ovvero un caso particolare di retroazione dallo stato.

%Si può ulteriormente caratterizzare il polinomio $P1$ per il caso di
%specie. Volendo analizzare modi $m=1$ lo studio può essere eseguito con
%l'ausilio del solo campo radiale. I valori di campo e corrente si
%riducono al caso $(m=1,n)$.

 E' possibile inoltre semplificare la trattazione, condensando i termini
di forma $F\mn$ e $S\mn$ all'interno della mutua induttanza, e definendo
convenientemente un nuovo paramentro che riassume l'intero termine
residuo $R\mn$ \cite{gregoratto} della funzione di trasferimento:
\begin{equation}
 {P_1}\mn = \sum_{m',n'} \frac{R\mnp}{(s-\gamma\mnp)}
\end{equation}
Tutti i poli con tassi di crescita negativi sono instabili e devono
essere controllati.  La somma in $m' = m+lM, n'= n+pN$ ha evidentemente
infiniti termini ma fortunatamente il fattore di forma, come si è
precedentemente mostrato, agisce come un attenuatore per le armoniche
elevate.  Nel contempo sappiamo che anche il plasma si mostra via via
più stabile per modi elevati; é quindi logico pensare di limitare il
numero di addendi alle prime armoniche identificando quali siano i poli
instabili.  Nel procedimento qui adottato, dopo aver ordinato le coppie
zero-polo rispetto alla stabilità del polo, si applica un ulteriore
troncamento del numero di poli per ottenere una utile riduzione della
dimensione del sistema a cui applicare il guadagno.
\begin{align}
 \label{eq:trasferimento}
 {P_1}\mn &= \sum_{j,k} \frac{R\mnp}{s-\gamma\mnp} =
 \frac{b_0+b_1s+\hdots+b_{n-1}s^{n-1}}{a_0+a_1s+\hdots+s^n}
\end{align}
La teoria della realizzazione asserisce che una funzione matriciale
strettamente propria ha sempre soluzione in un sistema in forma canonica
di controllo e una in forma canonica di osservazione.  Volendo indagare
una sigola coppia di modi $(m,n)$ è altresì facilmente dimostrabile che
il sistema in forma canonica di controllo, con le matrici
\newcommand{\canonicalcontrolA}{ 
\left[
\begin{matrix}
 0 & 1 && \\
 &&\ddots \\ \\
 &&&1 \\
 -a_0 & -a_1 & \hdots & -a_{n-1}
\end{matrix}
\right]
}
\newcommand{\canonicalcontrolB}{ 
\left[
\begin{matrix}
0\\ 0\\ \vdots \\ \\ 1
\end{matrix}
\right]
}
\newcommand{\canonicalcontrolC}{ 
\left[
\begin{matrix}
B_0& b_1& \hdots & & b_{n-1}
\end{matrix}
\right]
}
\begin{align}
 A &= \canonicalcontrolA,\quad B = \canonicalcontrolB \\ \nonumber
 \\ C &= \canonicalcontrolC \nonumber
\end{align}
realizza la (\ref{eq:trasferimento}). Infatti si ottiene immediatamente
la funzione di trasferimento con $y(t)/u(t) = C(sI-A)^{-1}B$. Avendo
scelto la realizzazione di una singola coppia $(m,n)$ è anche possibile
dire che essendo i poli tutti separati tra loro numeratore e
denominatore della (\ref{eq:trasferimento}) sono primi tra loro e la
realizzazione è minima. In questo caso gli autovalori della matrice di
stato coincidono con i poli della funzione di trasferimento e sono tutti
allocabili.

%Si considera il tasso di crescita della camera a vuoto con il valore
%definito da C.~G.~Gimblett in \cite{pizz_96} utilizzato nei precedenti
%articoli di R.~Fitzpatrick \cite{pizz_70} e R.~Paccagnella e P.~Zanca
%\cite{pizz_97}.
