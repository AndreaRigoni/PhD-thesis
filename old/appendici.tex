 \section{Operatori differenziali nel sistema cilindrico}

Nelle coordinate curvilinee del sistema di riferimento cilindrico gli
operatori di gradiente, divergenza e rotore assumono la seguente forma:

\begin{equation}
 \nabla f = \diffr{f} ; \frac1{r} \diff{\theta}{f} ; \diff{z}{f}
\end{equation}

\begin{align}
 \nabla \cdot \vettore{v} &= \frac1{r}\left( \diffr{(v_rr)} +
				      \diff{\theta}{v_\theta} +
 \diff{z}{rv_z} \right) \nonumber \\
&= \frac1{r}\left(\diffr{(rv_r)}\right) +
 \frac1{r}\diff{\theta}{v_\theta} + \diff{z}{v_z}
\end{align}

\begin{align}
 \nabla \times \vettore{v} &= \frac1{r} \left| 
 \begin{matrix}
  \hat u_r & \hat u_\theta & \hat u_z \\
  \diffr{} & \diff{\theta}{} & \diff{z}{} \\
  v_r & rv_\theta & v_z
 \end{matrix} \nonumber
\right| \\ &= \frac1{r}\left( \diff{\theta}{v_z} - \diff{z}{rv_\theta}
 \right) ; \left( \diff{z}{v_r} - \diffr{v_z}\right) ; \frac1{r}\left(
 \diff{\theta}{v_r} - \diffr{rv_\theta} \right)
\end{align}

considerando il sistema cilindrico come la linearizzazione della
geometria toroidale, con raggio del toro $R_0$, si introduce la
sostituzione $dz = R_0d\varphi$.