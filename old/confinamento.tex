%\section{Il confinamento magnetico di RFX}

\subsection{Le instabilità magnetoidrodinamiche in RFP}
- descrizione magnetoidrodinamica di singolo fluido
- modello magnetostatico
- tipi di instabilità
\newline

Sulla riuscita dei sistemi di contenimento elettromagnetico incidono i
fenomeni di cosidetta instabilità del plasma.

In esperimenti di tipo Tokamak e RFP, le instabilità prodittive di
effetti macroscopici sono descritte, nella loro forma più semplice, dal
modello fluido di plasma MHD (magnetohydrodynamic model). Dato il loro
carattere periodico, queste instabilità sono ben rappresentate da
componenti modali della forma $\exp i(m\theta + n\phi)$, soprattutto in
presenza di un elevato rapporto d'aspetto ed una sezione della camera a
vuoto circolare.

Una volta considerata ogni perturbazione nello spettro delle frequenze spaziali
lungo la colonna di plasma, è con la variazione o la curvatura del campo
che si cerca di riportare il sistema all'equilibrio.
Si noti fin d'ora, tuttavia, che il tentativo di stabilizzazione è vano
se eseguito lungo le superfici magnetiche per le quali i modi
perturbanti risuonano con il passo del fattore di sicurezza pari a
$q=m/n$\cite{wesson}.

\subsubsection{Il modello magnetoidrodinamico}
Tipicamente l'analisi di plasma fluido coinvolge i movimenti
generalizzati di ogni specie ionica, tuttavia è possibile introdurre una
semplificazione condensando in un unico parametro ogni coppia di valori
relativa a ioni ed elettroni, ottenendo un andamento di singolo fluido.
Le grandezze in gioco sono:


\begin{align}
 & \rho_m = n_e m_e + n_i m_i + n_0 m_0 
 && \rho = n_e q_e + n_i q_i
 \\ 
 & \media{J}_m = n_e m_e \media{v}_e + n_i m_i \media{v}_i + n_0 m_0
 \media{v}_0 
 && \media{J} = n_e q_e \media{v}_e + n_i q_i \media{v}_i
 \\
 & \media{v} = \frac{\media{J}_m}{\rho_m} 
 && p \approx n_ekT_e + n_ikT_i
\end{align}

Con queste variabili si possono definire per il fluido le equazioni di
continuità di massa e momento:



\begin{align}
\label{eq:mhd_mass_continuity}
\difft{\rho} + \nabla \cdot (\rho \vettore{v}) &= 0 \\ 
\label{eq:mhd_moment_continuity}
\rho \left(\difft{\vettore{v}} + \vettore{v} \cdot \nabla \vettore{v} \right) &=
\sigma \vettore{E} + \vettore{j} \times \vettore{B} - \nabla p \\
\label{eq:mhd_ohm}
\vettore{E} + \vettore{v} \times \vettore{B} &= \eta \vettore{j} 
%-\frac{ \vettore{j} \times \vettore{B} - \nabla p_e }{ne}
\end{align}

a cui si aggiungono i vincoli di interazione elettromagnetica posti
dalle leggi di Maxwell.
\begin{align}
\nabla \times \vettore{E} &= -\difft{\vettore{B}} \\
\label{eq:ampere}
\nabla \times \vettore{B} &= \mu_0 \vettore{J} \\
\nabla \cdot \vettore{B} &= 0
\end{align}

Il sistema così composto, detto modello \emph{MHD resistivo} del plasma,
risulta non chiuso; sono quindi necessarie ulteriori posizioni
semplificative:

\begin{itemize}
\item si considera la neutralità della carica $\sigma_i \simeq \sigma_e$
      che comporta l'annullamento del termine di resistività $\rho = 0$
      (il plasma è infatti studiato ad una distanza molto superiore alla
      lunghezza di Debye, così da apparire elettricamente neutro);
\item
     si aggiunge un vincolo di stato per il gas ideale, considerando solo
     trasformazioni isoterme o adiabatiche:
     \begin{equation}
     \difft{} \left( \frac{p}{\rho_m^\gamma} \right) = 0
     \end{equation}
     \begin{align}
      & \gamma = 1 && isoterme \\
      & \gamma = 5/3 && adiabatiche
     \end{align}
\end{itemize}

Fatto questo, un primo modo di ottenere una soluzione chiusa del sistema
è inserire l'equazione di Ampere (\ref{eq:ampere}) in quella di
Navier-Stokes della continuità del momento (\ref{eq:ampere}), riducendo
il sistema ad una equazione che descrive il comportamento della
pressione e tensione magnetica del plasma
\begin{equation}
 \label{eq:mhd_sol1}
 -\nabla \left( p+\frac{B^2}{2\mu_0} \right) + \frac{
  (\vettore{B}\cdot\nabla)B}{\mu_0} = 0
\end{equation}
Una seconda soluzione prevede invece l'inserimento dell'equazione di
Ampere in quella di Ohm (\ref{eq:mhd_ohm}), ottenendo la variazione del
campo di induzione che risulta come composizione di un termine di flusso
e un termine di diffusione
\begin{equation}
\label{eq:mhd_sol2}
 \difft{\vettore{B}} = \nabla \times (\vettore{v} \times \vettore{B}) +
  \frac{\eta}{\mu_0} \nabla^2\vettore{B}
\end{equation}

Se consideriamo un esempio semplice di pinch lineare, in cui la
componente di tensione magnetica sia nulla\footnote{la tensione
magnetica è infatti espressione della curvatura delle linee di campo},
si può notare come l'effetto di strizione della (\ref{eq:mhd_sol1})
dipenda dal gradiente di campo della colonna di plasma. Un'importante
figura di merito del confinamento magnetico è, infatti, espressa da
$\beta$, un parametro adimensionale che esprime la pressione cinetica
$p$ rispetto alla pressione magnetica $p_{mag} = B^2/2\mu_0$ e che per
il pinch lineare viene a dipendere proprio dal rapporto tra campo
esterno e campo penetrato:
\begin{equation}
\beta = \frac{p}{p_{mag}} = 1-\frac{B_{int}}{B_{ext}}
\end{equation}

Dal rapporto dei termini di (\ref{eq:mhd_sol2}), considerando come
termine di viscosità $\nu_m=\eta/\mu_0$, si ottiene il \emph{Numero di
Raynolds magnetico} $\mathcal{R}_m$, parametro caratteristico del moto
del fluido:
\begin{equation}
 \frac{\vert \nabla\times\vettore{v}\times\vettore{B}\vert}
  {\vert \nu_m\nabla^2\vettore{B}\vert} \simeq
  \frac{ \frac{v \cdot B}{L} }{ \nu_m \frac{B}{L^2}} = 
  \frac{vL}{\nu_m} \equiv \mathcal{R}_m
\end{equation}

Per ottenere alte pressioni cinetiche è desiderabile avere un plasma con
una ridotta penetrazione e quindi un'esigua viscosità magnetica
($\mathcal{R}_m\gg1$), ma nel contempo, però, si ottiene un moto
turbolento in cui la variazione di campo dipende in misura sempre
maggiore da fenomeni di trasporto.  E' per questo motivo che il modello
di studio per le proprietà di stabilità e per gli effetti della
geometria magnetica sullo stato di equilibio del plasma è ricavato
nell'ipotesi di perfetta conduttività $\eta = 0$ ed è chiamato
\emph{modello MHD ideale}\footnote{Per un'analisi dettagliata delle
condizioni del modello MHD ideale di veda\cite{fridberg}\cite{pizz27}}.

\subsubsection{Equilibrio Statico}
Lo studio dell'instabilità MHD del plasma riguarda principalmente le
perturbazioni del sistema ideale a partire da un punto di equilibrio
magnetostatico. In questo stato le relazioni che presentano una
variazione temporale sono evidentemente nulle, così il sistema,
imponendo $\eta = 0$ e $\difft{}=0$, divene:

\begin{equation}
 \label{mhd_ideal}
 \left\{
  \begin{array}{l}
   \nabla \cdot \vettore{B} = 0 \\
   \nabla \times \vettore{B} = \mu_0 \vettore{J} \\
   -\nabla p + \vettore{J} \times \vettore{B} = 0
  \end{array}
 \right.
\end{equation}
Dalla terza relazione il gradiente di pressione rimane ortogonale alle
linee di campo e corrente, costruendo un insieme di superfici isobare
toroidali\footnote{Si noti che i campi magnetico e di corrente della
prima e seconda relazione sono solenoidali e conducono a superfici
necessariamente chiuse. Se si considera un modulo di pressione costante
lungo detta superficie, con angolo di intersezione tra i vettori di
campo perpendicolare ovunque, si ottiene come soluzione reale possibile
il toro.}, una interna all'altra, chiamate \emph{superfici
magnetiche}\cite{boydsand}.
%\begin{figure}[th]
% \centering
 % \subfigure[gaussiana incorrelata]{images/sup_isabare.jpg}
% \includegraphics[width=5cm , bb= 20 20 575 575]{images/supisobare.jpg} 
%\end{figure}
Queste sono dette \emph{superfici razionali} quando le linee di campo
che le percorrono si ricombinano su se stesse dopo alcuni giri
toroidali, oppure \emph{ergodiche} quando, non ricombinandosi, coprono
l'intera superficie.

Nonostante la struttura fisica toroidale della macchina reale, è
possibile considerare un'approssimazione cilindrica (detta
assialsimmetrica) per semplificare la trattazione delle forze in gioco e
rendere il problema più accessibile, imponendo idealmente di considerare
periodiche anche le grandezze lungo l'asse longitudinale del
cilindro. In questa particolare geometria, che sarà usata anche nel
presente modello di studio, si possono individuare due configurazioni di
corrente che realizzano l'effetto \emph{''pinch''} del plasma
\cite{fridberg}\cite{ortolani} rispettivamente chiamate
$\theta$-\emph{pinch} e $z$-\emph{pinch}, a seconda della direzione
della corrente stessa\footnote{Ovvero lungo la direzione angolare e
assiale del cilindro come rispettive trasformazioni delle coordinate
poloidale e toroidale.}. Ogni configurazione possiede vantaggi e
svantaggi e le soluzioni proposte tendono a trovare il giusto
compromesso nel miscelare le due azioni senza sacrificare la stabilità
che garantisce il tempo di confinamento\cite{boydsand}.  Utilizzando
questo tipo di approssimazione l'equilibrio toroidale è ben descritto
dal modello di Grad-Shafranov tramite un'equazione differenziale
parziale ellittica (GSE) che analizza la stabilità del sistema
(\ref{eq:mhd_ideal}) in termini di funzioni di flusso\cite{boidsand}; la
maggior parte delle configurazioni magnetiche oggi in uso, come i
Tokamak e gli stessi RFP sfruttano questo importante modello di studio.
Introducendo $m$ e $n$ come coordinate poloidale e toroidale del modo
($m,n$) nella trasformazione spaziale in serie di Fourer, possiamo
esprimere l'evoluzione della perturbazione $\tilde{\psi}$ della generica
funzione di flusso analizzata in GSE
\begin{equation}
 \tilde{\psi}(\vettore{r},t) =
  \sum_k{ \tilde{\psi}_k(r) e^{ i(\vettore{k} \cdot \vettore{r} - \omega
  t) } } =
  \sum_k{ \tilde{\psi}_k e^{ i(m\theta + n\varphi - \omega t) } }
\end{equation}
dove $\vettore{r} = (r,\theta,\varphi)$ è il vettore di posizione in
coordinate toroidali e $\vettore{k}$ è il vettore dei relativi numeri
d'onda. La frequenza è invece l'espressione della trasformata nel tempo
ed è in generale un valore complesso $\omega = \omega_R + i\omega_I$,
in cui la parte reale esprime la velocità di propagazione dell'onda e la
parte immaginaria rappresenta la crescita, smorzata $(\omega_I <0)$ o
esponenziale ($\omega_I >0$), dell'ampiezza della perturbazione.  Il passo
di tale perturbazione si ottiene quindi per $(m\theta + n\varphi) =
cost$, ovvero: $$ md\theta -nd\varphi = 0$$ $$ p_p =
\int_0^{\Delta\varphi}d\varphi = \int_0^{2\pi}\frac{m}{n}d\theta =
\frac{m}{n}2\pi$$ Allo stesso modo il passo delle linee di campo
equivale a: $$\frac{Rd\varphi}{rd\theta} = \frac{B_\varphi}{B_\theta}$$
$$p_b(r) = R\Delta\varphi =
\int_0^{2\pi}\frac{1}{R}\frac{B_\varphi(r)}{B_\theta(r)}rd\theta = 2\pi
r\frac{B_\varphi(r)}{B_\theta(r)}$$ La perturbazione $\tilde{\psi}(m,n)$
instabile è, secondo il modello, quella i cui passi sono accoppiati con
le linee di campo; a questo scopo è utile definire il \emph{''fattore di
sicurezza''} $q(r)$ come un unico parametro che rappresenta il valore
dei campi a cui i modi sono accoppiati
\begin{align}
 \label{sec_fact}
  p_b(r) = p_p && \Leftrightarrow &&
 \frac{m}{n} = \frac{r}{R}\frac{B_\varphi(r)}{B_\theta(r)} \equiv q(r) 
\end{align}

A loro volta più superfici adiacenti che presentano uguali fattori di
sicurezza si accoppiano tra loro dando origine a fenomeni di risonanza
secondo il criterio di Suidam (1958) e sono esse stesse fonte di
instabilità; per questo motivo si introduce il parametro detto
\emph{''shear''}
\begin{equation}
 s(r) \equiv \frac{r}{q(r)}\frac{dq(r)}{dr}
\end{equation}
Nella figura [xxx] sono graficati i profili radiali del fattore di
sicurezza per esperimenti Tokamak e RFP; si noti come la configurazione
sia scelta in modo tale da non cadere mai nel valore unitario, e gli
andamenti rappresentino funzioni strettamente monotone.

\subsubsection{classificazione delle instabilità MHD}
Le instabilità sono generalmente classificate in relazione al contesto
fisico considerato; per formulare in modo corretto il problema
magnetoidrodinamico si è visto come sia necessario specificare un insieme
di \emph{condizioni al contorno} che accoppiano il plasma con il campo
magnetico esternamente applicato. Le classi di perturbazioni che si
vogliono analizzare devono quindi rientrare nei vincoli imposti al
sistema per mantenere un significato fisico nell'esperimento reale.

Chiamiamo anzitutto $\tilde{\zeta}(a)=0 $ lo spostamento della colonna
di plasma $\tilde{\zeta}$ e la normale alla superficie non perturbata
$\vettore{n_0}$. Nel caso di \emph{modi interni} (o
\emph{fixed-boundary}) $\vettore{n_0}\cdot\tilde{\zeta}(a) = 0$ non è
necessario uno spostamento del plasma $(\delta W_s = \delta W_v =
0)$. Queste microinstabilità, infatti, non hanno effetti macroscopici
sulla superficie della colonna di plasma, lo degradano soltanto
alterando i coefficienti di trasporto.  I \emph{modi esterni} (o
\emph{free-boundary}) $\vettore{n_0}\cdot\tilde{\zeta}(a)\neq 0$
comportano, invece, lo spostamento dell'interfaccia vuoto-plasma.

Tra questi ultimi, i modi instabili \emph{``pressure driven''},
caratterizzati da numero poloidale $m=0$, sono dovuti a scambi di
flusso, quindi essenzialmente di tipo idrodinamico, e danno luogo ad
instabilità in genere denominate \emph{``interchanges
instabilities''}. Questo tipo di instabilità è facilmente descrivibile
considerando un esperimento z-\emph{pinch} puro in geometria cilindrica;
come visibile in figura [xxx] restringimenti assiali della colonna di
plasma comportano diversi valori di $B_\theta \sim 1/r$; la pressione
cinetica è ovunque la stessa mentre quella magnetica tende ad essere più
forte in corrispondenza della sella, originando rigonfiamenti che
vengono sempre più amplificati.

Una diversa configurazione detta \emph{kink}, illustrata in figura
[xxx], è invece formata da modi $m=1$ cd \emph{''current driven''}. In
questo caso la curvatura della colonna di plasma comporta l'addensarsi
del campo in corrispondenza del raggio di curvatura minore e, quindi, si
assiste anche in questo caso ad un aumento della pressione magnetica che
tende via via ad autoamplificarsi. Rispetto alle instabilità
\emph{pressure driven} le \emph{kink} sono molto pericolose in quanto
rapide e difficilmente controllabili, con tempi di crescita dell'ordine
di $1/\tau_A$\footnote{$\tau_A = a/v_A$, con $v_A$ velocità di Alfvén e
$a$ raggio minore del toro, è riferito a variazioni di campo congelato
al plasma\cite{fridberg} }; esse si formano per la presenza di superfici
risonanti e sono generalmente stabilizzate da un forte campo
toroidale. La condizione di \emph{Kruskal-Safranov}, infatti, impone
$q(a)>1$ come minimo limite per la stabilità delle \emph{kink}, anche se
generalmente per un tokamak si mantiene un rapporto tra campo toroidale
e poloidale tale da avere valori di $q(a)\geq3$. Diverso ovviamente è il
caso di RFP dove la stabilizzazione è un problema più complesso e
coinvolge la presenza di correnti indotte nella scocca della camera
toroidale e, come si dirà a breve, l'uso di controllo attivo.

\subsection{Le instabilità di parete (RWM)}

E' facile notare come in RFP, per mantenere monotono l'andamento del
\emph{fattore di sicurezza}, ovvero \emph{shear} diverso da zero, sia di
fatto necessario il rovesciamento del campo che quindi presenta valore
negativo (e non nullo) in prossimità della parete della scocca. Questa
deve permettere il recircolo della corrente indotta per fungere da
conservatore di flusso; a questo scopo è rivestita da un mantello in
rame che ne aumenta la conduttività.  I \emph{''resistive wall modes''}
(RWM) sono instabilità di tipo \emph{kink} che si formano per la (seppur
esigua) resistenza della parete della camera a vuoto; sono relativamente
lenti poichè dipendono dal tempo di penetrazione del campo nel materiale
del mantello $\gamma_{RWM} \propto 1/\tau_w$.  Questo valore è definito
come:
\begin{equation}
 \tau_w = \mu_0 \sigma r_w \delta_w
\end{equation}
in cui $\sigma$ la conduttività della parete, $r_w$ il suo raggio e
$\delta_w \ll r_w$ il suo spessore. Le RWM sono inoltre caratterizzate
dalla presenza di campo radiale $b_r|_{r=r_s} \neq 0$ (dove $r_s$ è il
raggio relativo alla superficie risonante che genera il modo instabile),
aspetto che rende possibile la ricombinazione delle linee di campo.

La crescita spontanea di instabilità RWM è stata sperimentalmente
osservata in RFP con modi non risonanti $m=1$
\cite{pizz46}\cite{pizz47}\cite{pizz48} e modi
\emph{''pressure-driven''} con $n\neq 1$.  La loro stabilizzazione è
un'importante sfida, poichè esse sono tra i principali fattori che
limitano il raggiungimento di configurazioni ad alto $\beta$ per le
moderne macchine tokamak.

[forse inserire motivo instab TM con effetto RFA]


%\subsection{Il modello di Newcomb - damping rates}
%-


\subsection{L'uso delle saddle coils in RFXmod}

\begin{figure}[ht]
 \centering
 \includegraphics[ width=8cm ] {images/elemento-shell.jpg}
 \caption{sezione della camera toroidale}
\end{figure}

L'esperimento RFX-mod di Padova, come anticipato, consiste in una
macchina a rovesiamento di campo caratterizzata dai seguenti parametri
costruttivi: raggio interno della camera $a=0.459m$, raggio del asse
circolare del toro $R_0 = 1.995m$, corrente massima si scarica
$I\leq2MA$, e durata di aquisizione dell'impulso di $350ms$.  Le
modifiche principali apportate alla precedente versione, operativa nel
periodo 1992-1999, sono state finalizzate al miglioramento del profilo
``sharp-boudary'' del plasma e all'introduzione del controllo attivo dei
modi MHD.  La struttura meccanica della camera a vuoto è costuita in
Iconel 625, mentre all'interno una prima parete di 72x28 piastrelle di
graffite limitano la $Z_{eff}$ delle impurità libere nel plasma e una
seconda parete formata da un guscio (\emph{''shell''}) di rame di 3mm di
spessore ospita le correnti indotte dal campo interno. Il valore di
tempo di penetrazione per un campo verticale si assesta in $t_w = 50
\div 60 ms$ \cite{pizz75}

RFX-mod è equipaggiato con un sofisticato sistema di controllo attivo
composto da 192 bobine a sella (``\emph{saddle coils}'') organizzate in
4x12 array di 4 bobine ciascuno separatamente alimentati e localizzati
sulla superficie esterna della shell alla distanza (media) di $r_c =
0.513m$. I corrispondenti sensori magnetici in grado di rilevare il
campo radiale e toroidale sono posizionati all'interno della parete
conduttiva alla distanza di $r_s = 0.508m$.  Il sistema di amplificatori
è in grado di generare modi non assialsimmetrici m=0 e n < 6 e può
funzionare configurato in diverse modalità: ed esempio lo schema
chiamato \emph{''Virtual Shell''} (VS) è un sistema di retroazione
locale che, pur non tenendo conto degli effetti globali sul plasma, è
già in grado di ridurre fortemente la componente radiale di campo
$b_r$\cite{pizz78}\cite{pizz79}

Il metodo qui presentato aggiunge al precedente la ricostruzione
completa dell'andamento di detta componente, misurata dai 192 sensori,
introducendo un modello per l'attraversamento della parete e l'analisi
delle componenti di aliasing nelle armoniche più elevate.

