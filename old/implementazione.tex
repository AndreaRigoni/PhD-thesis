
\subsection{Sistema orientato agli oggetti}
Il raggiungimento dell'obiettivo comune e l'enorme quantità di fondi
necessari agli esperimenti pongono la ricerca sulla fusione
termonucleare controllata nel panorama scientifico come uno dei più
vasti ambiti di cooperazione internazionale.  I recenti esperimenti JET
e ITER sono infatti frutto dello sforzo congiunto di molti paesi
consorziati e rappresentano un chiaro esempio di come il lavoro
combinato possa dare una forte accelerazione verso la concretizzazione
dei risultati; dal punto di vista economico agevolati dalla divisione
delle spese e dal punto di vista scientifico arricchiti per la
possibilità di attingere ad esperienze specilistiche maturate nel tempo
dagli stati membri. Per questa ragione è auspicabile se non necessario
orientare lo sforzo di ricerca verso un sistema integrato e standardizzato.

Il modello di studio presentato nel precedente capitolo ha avuto
precedenti implementazioni in riferimento a diversi studi svolti sugli
esperimenti EXTRAP T2R \cite{gregoratto} e RFXmod stesso
\cite{pizzimenti}. Il codice numerico usato nei lavori citati è stato
quì adattato al funzionamento in un ambiente orientato agli oggetti e
riformulato per funzionare in una piattaforma a sorgente aperto.

La programmazione orientata agli oggetti (Object Oriented Programming) è
in generale un paradigma di programmazione, che prevede la definizione
di oggetti software dotati di attributi (dati), e metodi (procedure);
interagenti tra loro mediante scambio di messaggi. Tale scelta, vista
come un supporto alla modellazione software per l'oggetto di studio, è
volta ad ottenere un effettivo supporto naturale alla modularizzazione e
al riuso del codice. La modularità del codice garantisce una maggiore
chiarezza di lettura creando le basi per una piattaforma comune in cui
risulti uniformato l'approccio alle entità di studio.  Di fondamentale
importanza per capire il valore della programmazione ad oggetti é il
concetto di astrazione che storicamente ha visto via via un sempre
maggior sviluppo nel lessico informatico. Un concetto sviluppatosi
inizialmente all'interno della così detta programmazione procedurale che
per prima propose l'idea di interfaccia per nascondere i dettagli
dell'implementazione e la connessione con il livello hardware da ciò che
divenne la programmazione applicativa (tipicamente l'interfaccia che
viene dichiarata viene detta Application Programming
Interface). L'aspetto definitivo del modello di dato astratto fu
proposto tramite l'entità denominata classe, questa rappresenta
effettivamente una struttura dati corredata dal codice per gestirla ma
aggiunge anche altre idee come eredità e polimorfismo. Un oggetto è
quindi una istanza di una classe, ossia la realizzazione di un dato
secondo le specifiche definite nella classe stessa. Matematicamente gli
oggetti, visti come istanze, sono i membri dell'insieme classe definito
in modo intensivo, ovvero attraverso le sue caratteristiche invece che
elencandone gli elementi; tali caratteristiche comprendono la
definizione dei tipi di dati degli attributi, l'accesso di questi da
membri esterni alla classe, e vincoli che la tupla degli attributi può
assumere. In questo senso l'oggetto assume la forma ben nota
nell'information theory di macchina a stati finiti.

Nel presente lavoro è stato anche analizzato il più recente inserimento
dell'uso degli oggetti all'interno della IPC (inter process
comunication), ovvero qulla branca della informatica che studia la
comunicazione tra diversi processi, nel caso di condivisione locale o
remota della memoria. La gran parte del processo di produzione del
software è stata rivolta alla possibilità di interfacciare il modello
con altre entità, esterne al contesto dell'esperimento stesso. Si è
voluta dare una prospettiva di un nuovo paradigma di studio dove sia
possibile e semplice l'integrazione del codice, troppo spesso scritto al
solo scopo di produrre una singola pubblicazione e fortemente legato ad
un contesto tanto specifico da renderlo inutilizzabile in un diverso
ambiente. Gli attributi degli oggetti divengono quindi sede dello
scambio di informazioni mentre i metodi possono diventare
implementazioni delle RPC (Remote Procedure Calls, procedure chiamate su
macchine remote). Recentemente si stà assistendo ad un sempre più
intensivo uso di questa tecnologie, più o meno raffinate, per accedere a
dati remoti o lanciare richieste di elaborazione. Basti pensare ad
esempio ad una richiesta ad un motore di ricerca per il web, dove un
dato testuale viene usato per istruire una procedura remota che analizza
un vastissimo database diffuso su migliaia di server per restituire una
pagina formattata. Il panorama offerto dal crescente accesso alla rete
globale ha mostrato l'affermarsi di varie tecnologie per l'interscambio
di oggetti software; inizialmente appannaggio di poche aziende dominanti
il mercato, e successivamente definite tramite standard aperti.  Nel
1991 venne rilasciata dal OMG (Object Management Group) la prima
versione del sistema CORBA (Common Object Request Broker Architecture),
realizzato con l'intento di unificare in un unica infrastruttura
client-server tutti i possibili linguaggi di programmazione.  CORBA di
fatto è in grado mettere in comunicazione frammenti di codice separati,
scritti tramite linguaggi diversi, e operanti di diversi computer, come
se funzionassero all'interno di una unica applicazione. Il meccanismo di
funzionamento è principalmente basato sulla idea di normalizzazione
della semantica nella definizione degli oggetti. Questo metodo è
realizzato da un ulteriore linguaggio chiamato IDL (Interface Definition
Language) per definire le interfacce che gli oggetti presentano
all'esterno; vi è poi un sistema di ``mappatura'' tra la interfaccia IDL
dichiarata e il linguaggio specifico. Esistono implementazioni di mappe
standard per i linguaggi: C, C++, Java , Lisp, Python, Ruby, Tcl e altri
\cite{url_corba}.  L'utilità di questo approccio e l'importanza che
riveste attualmente in ambito scientifico giustifica l'attenzione
rivolta alla scelta sulla metodologia di programmazione per questo
lavoro e in generale per lo sviluppo di modelli di studio in
RFX. Accanto alla interoperabilità del livello applicativo vi è anche un
crescente intaresse per una condivisione a livello di infrastruttura

[uso di grid, cloud computing - glite globus]

Il presente lavoro vuole proporre l'intera analisi numerica
attraverso codice a sorgente aperto.  Tale filosofia è stata peraltro
scelta nella distribuzione di MDSplus, il software sviluppato tramite la
collaborazione del Massachusetts Institute of Technology, il consorzio
RFX e il Los Alamos National Lab. Attualmente MDSplus si è affermato
come lo standard de facto per l'acquisizione e il mantenimento dei
segnali negli esperimenti di fusione e viene utilizzato da più di 30
installazioni dislocate su 4 continenti.


\subsection{L'interfaccia G-MDS}

Il sistema scelto è situato a cavallo tra la programmazione a basso
livello e i sistemi evoluti di astrazione; si chiama GObject ed è il
sistema di oggetti che costituiscono la base del noto ambiente desktop
GNOME (GNU Network Object Model Enviroment) \cite{url_gnome}, fortemente
orientato alla interoperabilità di rete. GObject è un framework creato
sulla libreria GLib nel contesto del progetto GTK+, una API per la
scrittura di programmi con interfaccia grafica. Anche se ancora
mantenuto all'interno del progetto GNOME la libreria di GObject è ora
completamente autonoma e se ne è reso possibile l'utilizzo
indipendentente dal suo iniziale scopo grafico. L'ambiente di base è
quindi la libreria GLib che grazie alla definizione di strutture dati
avanzate per la gestione della memoria, e un set di tipi primitivi, può
essere utilizzata come astrazione per la programmazione
multipiattaforma. Programmi che utilizzano i tipi derivati di GLib e
GObject possono, infatti, essere opportunamente compilati in ambienti
quali Windows, Mac OS e Linux/Unix derivati.

Anche se Gobject è in grado di fornire un sistema completo nella
programmazione ad oggetti per il C, il linguaggio per il quale è stato
ideato, esso è visto come una possibile alternativa ai linguaggi evoluti
C-derivati come C++ e Objective-C. la differenza fondamentale che lo
separa da questi è di non essere un linguaggio separato: GObject è
semplicemente una libreria e come tale non introduce una nuova sintassi
nel compilatore o preprocessore. Quindi, anche se l'insieme ``C e
GObject'' può essere considerato di fatto un linguaggio a sè stante,
esso è perfettamente compatibile e utilizzabile all'interno del C
standard. Librerie a basso livello, a stretto contatto con l'hadrware,
come sistemi digitali di acquisizione dati o processori di segnale,
possono così accedere alle strutture complesse senza programmi
intermedi. Inoltre la grande differenza che contraddistingue il sistema
GObject è l'enfasi che questo ripone nella gestione dei segnali (spesso
chiamata anche gestione degli eventi). Tale attenzione deriva
principalmente dal fatto che la libreria fu costruita sulla base delle
esigenze di un toolkit grafico; così che, mentre la gestione dei segnali
risulta generalmente un componente aggiuntivo di un linguaggio, in
GObject i segnali sono definiti all'interno del sistema. In questo modo
l'uso dei segnali nelle applicazioni che utilizzano GObject è più
immediata, aumentando l'utilizzo dell'incapsulamento degli oggetti e
rendendoli pronti all'uso distribuito, come vedremo in dettaglio di
seguito.


La struttura base di un oggetto GObject è delineata in figura 
%%%%%%%%%%%%%%%%%%%%%%%%%%%%%FIGURA%%%%%%%%%%%%%%%%%%%%%%%%%%%%%%%%%%%%%
\begin{figure}[!htb]
 \centering
% \includegraphics[ width=8cm ] {images/schema-software2.png}
 \caption{Diagramma della strittura di un oggetto GObject}
 \label{fig:gobject}
\end{figure}
%%%%%%%%%%%%%%%%%%%%%%%%%%%%%FIGURA%%%%%%%%%%%%%%%%%%%%%%%%%%%%%%%%%%%%%
Come esempio esplicativo presentiamo l'implementazione della classe base
per g-mds. La definizione nel file de header è la seguente:

% \begin{minipage}[c]{11cm}
  \includecodesample{code/g-mds-object.h}
% \end{minipage}

Mentre il file sorgente implementa i metodi ereditati dalla classe base:

% \begin{minipage}[c]{11cm}
  \includecodesample{code/g-mds-object.c}
% \end{minipage}

[spiegazione del funzionamento e introduziona alle properties]

Per quanto riguarda l'utilizzo nell'ambito dell'esperimento RFX sono
stati definiti vari oggetti base, da un lato estendendo la libreria GLib
per la gestione di matrici multidimensionali, dall'altro sviluppando
oggetti specifici per definire le entità principali.

%  \includecodesample{code/NDarray.c}

%%%%%%%%%%%%%%%%%%%%%%%%%%%%%FIGURA%%%%%%%%%%%%%%%%%%%%%%%%%%%%%%%%%%%%%
\begin{figure}[!htb]
 \centering
 \includegraphics[ width=8cm ] {images/albero-parentale.png}
 \caption{Diagramma parentale degli oggetti utilizzati in G-Mds}
 \label{fig:albero-g-mds}
\end{figure}
%%%%%%%%%%%%%%%%%%%%%%%%%%%%%FIGURA%%%%%%%%%%%%%%%%%%%%%%%%%%%%%%%%%%%%%


L'oggetto ``GMdsExperiment'' è preposto fondamentalmente alla
definizione dell'ambiente di lavoro: vi sono definite proprietà
specifiche quali raggio, rapporto di aspetto, geometria della camera,
ecc.. Gli oggetti di tipo contenitore servono invece a manipolare i dati
veri e propri organizzandoli opportunamente negli array precedentemente
definiti. Per fare, ad esempio, una indagine sulla griglia dei sensori
sarà sufficiente definire un ogetto GMdsGrid, agganciare alla struttura
una notazione e definire l'esperimento.

Per otterere il massimo della portabilità del codice, gli oggetti creati
vengono successivamente ridichiarati attraverso un apposito header che
li rende utilizzabili in C++. Di seguito è presentata la dichiarazione
per l'oggetto base.

\includecodesample{code/object.h}

La possibilità di ottenere un livello di comunicazione tra diversi
software di analisi, come preannunciato nel paragrafo precedente
parlando di IPC, avviene su due livelli. Anzitutto si vuole creare una
interfaccia per l'utilizzo del codice su qualsiasi linguaggio; si è
scelto per questo di utilizzare SWIG, un software molto noto per il
\emph{wrapping} del codice C, che rende utilizzabili le strutture dati
in ambiente Octave. Il wrapper agisce mediante una interfaccia creata
per la dichiarazione della struttura e per la definizione di opportuni
\emph{marshalers} per il passaggio dei tipi.  Un esempio è rappresentato
dalla ``typemap'' che è stata realizzata per convertire una matrice
Octave in un array tridimensionale per GMdsGrid:

\includecodesample{code/types.i}

Secondariamente si è posta l'attenzione alla introspezione diretta degli
oggetti che GObject recentemente ha introdotto nella
libreria~\cite{url_gobject_introspection}; questo sistema permette una
dichiarazione di interfaccia da parte degli oggetti stessi che sono
quindi in grado di rendersi autonomamente disponibili a diversi
linguaggi. L'importanza di questo secondo sistema è la possibilità di
autodichiarare gli oggetti nel sistema di interconnessione chiamato
D-Bus. Il progetto GNOME attualmente mantiene due sistemi per la
gestione della IPC: Il sistema Bonobo, sviluppato per gestire
l'interfacciamento ORBit (CORBA compatibile), e D-Bus che si pone come
valida alternativa per semplicità di implementazione. Con il proposito
di generare un codice di semplice utilizzo e per la buona integrazione
con la libreria Glib è stato scelto l'utilizzo del secondo, in vista poi
di un annunciato progressivo abbandono dell'uso di ORBit in ambito
desktop. Anche se D-Bus sembra di fatto una scelta obbligata, visto
l'ampia compatibilità che propone, rimane la possibilità di sviluppare in
un secondo momento delle \emph{middleware} che mettano in comunicazione
oggetti D-Bus con i Brooker delle grosse reti di calcolatori, accedendo
alle risorse di condivisione e \emph{cloud-computing} offerti da
sistemi Grid come GLite~\cite{url_glite} e Globus~\cite{url_globus}.

%%%%%%%%%%%%%%%%%%%%%%%%%%%%%FIGURA%%%%%%%%%%%%%%%%%%%%%%%%%%%%%%%%%%%%%
\begin{figure}[!htb]
 \centering
 \includegraphics[ width=8cm ] {images/schema-software2.png}
 \caption{Diagramma della organizzazione del software}
\end{figure}
%%%%%%%%%%%%%%%%%%%%%%%%%%%%%FIGURA%%%%%%%%%%%%%%%%%%%%%%%%%%%%%%%%%%%%%


\subsection{La costruzione del modello B.G.P. }

Riguardando brevemente le possibilità introdotte dal sistema delineato
dal paragrafo precedente, si descrive ora come queste si applichino in
pratica al codice del modello. Anzitutto, per aderire alla scelta di
mantenere per quanto possibile il software su piattaforma a codice
aperto, si è trasportato il sorgente dalla originaria esecuzione su
ambiente Matlab verso il linguaggio open-source per il calcolo numerico
GNU Octave \cite{url_octave}. L'accesso diretto ai dati di MdsPlus,
fornito a Matlab dalla interfaccia \emph{mdsvalue}, è un codice
Matlab-eseguibile (mex) presente nel corredo di utilità fornite da
MdsPlus stesso. L'eseguibile che viene compilato agisce tramite la
libreria C a basso livello chiamata \emph{Mdslib}, questa sarà, come
sarà presentato a breve, il medesimo punto di accesso a MdsPlus
utilizzato dal sistema ad oggetti presentato. La ricompilazione dei
sorgenti rende possibile l'utilizzo della interfaccia diretta anche in
ambiente Octave, si utilizza infatti il tool \emph{mkoctfile} che è in
grado di tradurre il codice per la compilazione di eseguibili di tipo
\emph{mex}.  Chiaramente la chiamata alla funzione di libreria per
l'accesso ai dati non è più realizzata da una specifica funzione Matlab
ma viene resa disponibile attraverso i metodi degli oggetti di G-Mds. In
questo caso, come precedentemente illustrato, i metodi sono inscatolati
dentro oggetti Octave mediante wrapper generati da SWIG, codice
egualmente compilato ricorrendo a \emph{mkoctfile} ma con il link
dinamico nativo (il sistema di oct-files). Nella stesura del programma
si potrà quindi fare riferimento a oggetti di tipo segnale,
instanziandoli mediante una etichetta che ne identifica la richiesta al
database; vi sarà poi un conseguente metodo get che effettua l'accesso
vero e proprio in relazione ai periodo temporale di studio e
all'intervallo di ricampionamento. In questo modo non è più necessario
definire ogni singolo parametro durante l'accesso ma basterà
caratterizzare via via il contesto dell'esperimento stesso attraverso
appositi oggetti.  E' stato costruito quindi un oggetto notazione, esso
di fatto rappresenta un funzionale che associa una generica tupla di
numeri interi ad una specifica etichetta. Come è facile intuire, tale
sequenza di numeri è usata per individuare facilmente una coordinata in
uno spazio discreto; si definisce infatti l'oggetto griglia come
sequenza ordinata di segnali eventualmente distribuiti a loro volta
nello spazio con specifiche coordinate.  In questo modo è possibile
utilizzare una semplice funzione Matlab o Fortran per istruire il
funzionale e utilizzare poi la notazione una volta per tutte in qualsiasi
linguaggio e in qualsiasi elaboratore accedendo all'oggetto istanziato.

Per quanto riguarda il controllo dei modi MHD si sono utilizzate
principalmente le notazioni che fanno capo ai segnali relativi alle
\emph{saddle coils}. Motriamo di seguito la definizione delle funzioni
che ricavano la lettura del campo radiale e la corrente dei generatori
collegati.

\begin{figure}[!htb]
 \footnotesize 
 \centering
 \begin{minipage}[c]{11cm}
  \includecodesample{code/notation.m}
 \end{minipage}
 \normalsize
 \label{fig:notation}
 \caption{Funzioni per la generazione della notazione}
\end{figure}
%\verbatiminput{code/notation.m}

Un'altra caratteristica introdotta è la definizione dei parametri
dell'esperimento attraverso proprietà specifiche degli oggetti. Viene
istanziato un oggetto esperimento in grado di mantenere i valori dei
parametri fisici e geometrici caratterizzanti la macchina; appare
evidente come questo comporti la possibilità di spostare facilmente il
codice da un esperimento all'altro, o tra diversi modelli, semplicemente
cambiando l'istanza dell'oggetto. E' semplice immaginare come questo
assuma grande utilità pratica di gestione del codice, soprattutto in
vista della applicazione di database ad oggetti in cui siano archiviate
e condivise le istanze che fanno capo ai modelli di uso comune. Questa
possibilità suggerisce ancora una volta l'importanza della uniformità di
linguaggio che l'uso delle interfacce introduce, rendendo più leggibile
la struttura del codice.

Le proprietà menzionate sono successivamente utilizzate all'interno dei
blocchi funzionali. Il blocco principale che riassume l'intera catena
della \emph{matching procedure} è ovviamente la definizione della
funzione di trasferimento del sistema. E' utile richiamare la
(\ref{eq:transf_s}) che rappresenta la trasformata nella formulazione
implementata:
\begin{equation}
{P_1}\mn = \sum_{m',n'} \frac{R\mnp}{(s-\gamma\mnp)} \nonumber
\end{equation}
Si calcola numeratore e denominatore per ongni armonica $(m',n')$,
all'interno di un prestabilito intervallo finito di valori, mediante due
cicli for annidati. La \figurename~\ref{fig:fc1} mostra uno schema
semplificato del diagramma a blocchi della funzione. 

%%%%%%%%%%%%%%%%%%%%%%%%%%%%%FIGURA%%%%%%%%%%%%%%%%%%%%%%%%%%%%%%%%%%%%%
\begin{figure}[ht]
 \centering
 \includegraphics[ width=8cm ] {images/flowchart-transf1.png}
 \caption{Diagramma a blocchi del calcolo della funzione di
 trasferimento \label{fig:fc1}}
\end{figure}
%%%%%%%%%%%%%%%%%%%%%%%%%%%%%FIGURA%%%%%%%%%%%%%%%%%%%%%%%%%%%%%%%%%%%%%
Il calcolo procede definendo anzitutto dei valori limite per il numero
di armoniche considerate, tale valore dipende principalmente dalla
conoscenza sulla stabilità tipica dei modi e dalla forma del fattore di
sagoma, poichè esso determina lo smorzamento delle armoniche più
elevate. E' poi necessario verificare se si stia lavorando con la
presenza del plasma nella camera, in questo caso sono inseriti valori
misurati dei tassi di crescita dei modi instabili; viceversa negli spari
a vuoto si considera il tasso di crescita della camera con il valore
definito da C.~G.~Gimblett in \cite{pizz_96}.  Per ogni armonica viene
aggiuna una coppia polo-residuo ad una lista che sarà infine ordinata
per valori decrescenti del polo e ridotta nel numero massimo di coppie
cosiderate nella approssimazione.

Il risultato finale è comunque in forma di coppie polo-residuo per
formare la matrice di stato. Viene comunque utilizzato il pacchetto
della teoria del controllo per Octave che prevede la definizione diretta
del sistema, una volta calcolato numeratore e denominatore mediante la
funzione \emph{residue}. La procedura genera un sistema caratterizzato
dalle matrici in forma canonica di controllo, come indicato a
conclusione del precedente capitolo.